El inicio del procesamiento de lenguaje natural suele remontarse a 1950 cuando Alan Turing publicó el articulo 'Computing Machinery and Intelligence' donde propone lo que conocemos como el test de Turing, un criterio de inteligencia.
\\
Una de las primeras aplicaciónes del NLP ocurrió en 1954 e involucro la traducción de 60 oraciones rusas al inglés.
\\
En los 1960´s ELIZA y SHRDLU fueron sistemas capaces de comunicarce con  humanos.
\\
Existen al menos 4 algoritmos de parsing:
\begin{enumerate}
    \item El algoritmo CYK (Cocke–Younger–Kasami)
    \begin{enumerate}
        \item Solo trabaja con CFG en la forma normal de chumsky
        \item El peor de los casos es $O(n^3 . |G|)$ time. Donde $n$ es el tamaño de la cadena y $|G|$ es el tamaño de la gramática $G$ lo que lo hace uno de los mas eficientes en el peor de los casos.
    \end{enumerate}
    \item Earley parser
    \begin{enumerate}
        \item Trabaja con todas las gramáticas libres del contexto.
        \item Tiene un tiempo de ejecución $O(n^3)$ en el caso promedio,$O(n^2)$ en gramáticas no ambiguas y $O(n)$ para CFG deterministas.
    \end{enumerate}
    \item LR parser (Left-to-right, Rightmost derivation in reverse)\\
    Son tipos de bottom-up parsers que analizan DCFG en tiempo lineal.
    \begin{enumerate}
        \item LALR parser
        \item Canonical LR parser
        \item Minimal LR parser 
        \item GLR parser
    \end{enumerate}
    
    \item LL parser (Left-to-right, Leftmost derivation)\\
    Es un top-down parser para un subset de CFG
\end{enumerate}